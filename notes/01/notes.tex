\documentclass{article}
\usepackage{amsmath}
\usepackage{verbatim}
\usepackage{nameref}
\usepackage{minted}
\usemintedstyle{tango}

\title{Week 1 Notes (Introduction to the Module)}

\begin{document}

\maketitle

\section{Lecture Portion}

\subsection{Motivation}

One of the main reasons why it's important to discuss logic in the
context of computer science is that it is very much possible (and
often necessary) to be 100\% certain that a program will do what
you want it to do.

There are some limitation to this, most notably the halting problem.

\subsection{Notation}

A statement over another statement means that the upper statement
implies the lower statement. These may be nested. 

eg.
\[
\frac{\text{All men are mortal, Socrates is a man}}
     {\text{Socrates is mortal}}
\]

\subsection{Types of Logic}

\subsubsection{Aristotelian}

Deals entirely with classes and belonging. Confusingly "X belongs
to Y" means that the \textit{classhood} of X belongs to Y, not that
X belongs to the set Y. For example, the statement X belongs to all
Y makes Y a subset of X.

There are four types of class relation:

\begin{itemize}
	\item a: belongs to every
	\item e: belongs to no
	\item i: belongs to some
	\item o: does not belong to some
\end{itemize}

This results in a series of syllogisms:

\begin{center}
	\begin{minipage}{0.3\linewidth}
	\begin{equation}
		\frac{AaB,BaC}{AaC}
	\end{equation}
	\end{minipage}
	\begin{minipage}{0.3\linewidth}
	\begin{equation}
		\frac{AeB,BaC}{AeC}
	\end{equation}
	\end{minipage}
	\begin{minipage}{0.3\linewidth}
	\begin{equation}
		\frac{AaB,BiC}{AiC}
	\end{equation}
	\end{minipage}
\end{center}

\begin{center}
	\begin{minipage}{0.3\linewidth}
	\begin{equation}
		\frac{AeB,BiC}{AoC}
	\end{equation}
	\end{minipage}
	\begin{minipage}{0.3\linewidth}
	\begin{equation}
		\frac{MaN,MeX}{NeX}
	\end{equation}
	\end{minipage}
	\begin{minipage}{0.3\linewidth}
	\begin{equation}
		\frac{MeN,MaX}{NeX}
	\end{equation}
	\end{minipage}
\end{center}

\begin{center}
	\begin{minipage}{0.3\linewidth}
	\begin{equation}
		\frac{MeN,MiX}{NoX}
	\end{equation}
	\end{minipage}
	\begin{minipage}{0.3\linewidth}
	\begin{equation}
		\frac{MaN,MoX}{NoX}
	\end{equation}
	\end{minipage}
	\begin{minipage}{0.3\linewidth}
	\begin{equation}
		\frac{PaS,RaS}{PiR}
	\end{equation}
	\end{minipage}
\end{center}

\begin{center}
	\begin{minipage}{0.3\linewidth}
	\begin{equation}
		\frac{PeS,RaS}{PoR}
	\end{equation}
	\end{minipage}
	\begin{minipage}{0.3\linewidth}
	\begin{equation}
		\frac{PiS,RaS}{PiR}
	\end{equation}
	\end{minipage}
	\begin{minipage}{0.3\linewidth}
	\begin{equation}
		\frac{PaS,RiS}{PiR}
	\end{equation}
	\end{minipage}
\end{center}

\begin{center}
	\begin{minipage}{0.3\linewidth}
	\begin{equation}
		\frac{PoS,RaS}{PoR}
	\end{equation}
	\end{minipage}
	\begin{minipage}{0.3\linewidth}
	\begin{equation}
		\frac{PeS,RiS}{PoR}
	\end{equation}
	\end{minipage}
\end{center}

\subsubsection{Boolean}

Boolean logic uses not ($\neg$), and ($\wedge$) and or ($\vee$).

For example:

\begin{center}
	\begin{minipage}{0.3\linewidth}
	\begin{displaymath}
		\frac{}{\neg(x \wedge \neg x)}
	\end{displaymath}
	\end{minipage}
	\begin{minipage}{0.3\linewidth}
	\begin{displaymath}
		\frac{x \vee y \qquad \neg x}{y}
	\end{displaymath}
	\end{minipage}
\end{center}

This makes logic behave somewhat like arithmetic.
Or ($\vee$) works a bit like addition ($+$),
while and ($\wedge$) works something like muliplication ($\times$).

There is no sense of "any", "some" or "none". Everything is taken
to be known as either true or false, so translating statements into
their equivalent in boolean logic can be hard in some cases.

\subsubsection{Frege's}

Frege extended Boole's logic to include "all" and "some".

Frege's notation is unpopular and people tend to use a notation
devised by Gentzen.

We use $\forall$ to mean for all, and $\exists$ to mean for some.

For example:

\begin{center}
	\begin{minipage}{0.3\linewidth}
	\begin{displaymath}
		\frac{\forall x . \Phi}{\Phi [t/x]}
	\end{displaymath}
	\end{minipage}
	\begin{minipage}{0.3\linewidth}
	\begin{displaymath}
		\frac{\Phi [t/x]}{\exists x . \Phi}
	\end{displaymath}
	\end{minipage}
\end{center}

The first means that if $x.\Phi$ holds for all $x$,
replacing $x$ with $t$ will also hold.
The second means that if replacing $x$ with $t$ holds, then $x.\Phi$
holds for at least some $x$.

The symbol $\Phi$ represents a string and $x$ represents a substring location.

\section{Tasks}

\subsection{}

\subsubsection{Make up an example per syllogism}

\paragraph{$\frac{AaB,BaC}{AaC}$}
Mortality belongs to all men, manhood belongs to all Socrates:
therefore mortality belongs to all Socrates.

\paragraph{$\frac{AeB,BaC}{AeC}$}
No dog is green, all poodles are dogs: therefore no poodle is green.

\paragraph{$\frac{AaB,BiC}{AiC}$}
All women are mortal, some people are women: therefore some people are mortal.

\paragraph{$\frac{AeB,BiC}{AoC}$}
No apes have tails, some primates are apes:
therefore not all primates have tails.

\paragraph{$\frac{MaN,MeX}{NeX}$}
All buckets are concave, no inflated footballs are concave:
therefore no inflated footballs are buckets.

\paragraph{$\frac{MeN,MaX}{NeX}$}
No coconuts are purple, all ripe plums are purple:
therefore no ripe plums are coconuts.

\paragraph{$\frac{MeN,MiX}{NoX}$}
No diamonds are conductive, some forms of carbon are conductive:
therefore not all forms of carbon are diamonds.

\paragraph{$\frac{MaN,MoX}{NoX}$}
All cakes are sweet, not all foods are sweet:
therefore not all foods are cake.

\paragraph{$\frac{PaS,RaS}{PiR}$}
All cats are silly, all cats are carnivores:
therefore some carnivores are silly.

\paragraph{$\frac{PeS,RaS}{PoR}$}
No planets are undergoing fusion, all planets are celestial bodies:
therefore not all celestial bodies are undergoing fusion.

\paragraph{$\frac{PiS,RaS}{PiR}$}
Some dogs are loyal, all dogs have bones:
therefore some things with bones are loyal.

\paragraph{$\frac{PaS,RiS}{PiR}$}
All whales have fins, some whales are blue:
therefore some blue things have fins.

\paragraph{$\frac{PoS,RaS}{PoR}$}
Not all cleaning products are bleach, all cleaning products are chemicals:
therefore not all chemicals are bleach.

\paragraph{$\frac{PeS,RiS}{PoR}$}
No monkeys can fly, some monkeys like bananas:
therefore not everything that likes bananas can fly.

\subsubsection{
	Explain the famous bug in Aristotle's logic,
	considering $\frac{PaS,RaS}{PiR}$,
	with $S$ = unicorns, $P$ = pink and $R$ = fluffy
}

Translating this into standard English gives:

All unicorns are pink, all unicorns are fluffy:
therefore some fluffy things are pink.

Of course, this doesn't actually hold because it assumes that at least some
unicorns exists, which isn't necessarily the case.

\subsubsection{Find a written political argument, map it to
premises and conclusions, and then discuss validity and soundness}

For this we'll be examining the statement
"It's literally giving somebody money for nothing"
from the Guardian headline:
"'It's literally giving somebody money for nothing':
the battle to reform property leaseholds" \cite{guardianarticle}

Premises:
\begin{enumerate}
	\item Money is being recieved
	\item The thing being given is nothing
\end{enumerate}

 

\subsection{}

\subsubsection{Attempt to create some given words from the MU
system, explaining why not if not}
Thoughts while attempting these:

\begin{itemize}
	\item You can make strings of I's which have length $2^n$.
	\item You can remove any string of I's which have length $6n$.
	\item You can multiply any string of I's by $2^n$.
	\item It feels impossible to get any odd number of I's.
	\item No power of 2 is a multiple of 6.
	\item Working backwards might help.
	\item I think I was over-abstracting in my head.
		We can make use of IaddU to get odd numbers.
		I was thinking of series of $6n$ I's as though they were
		already deleting themselves or as III as though it was
		already a U - but it isn't and the IaddU rules comes in
		very handy.
	\item This is best summarised as
		"You can always remove III at the end", or
		$\frac{\text{xIII}}{\text{x}}$.
	\item In order to get MU, we may need an odd number of U's.
		This would require an odd multiple of 3 of I's.
		So a number like 3, 9, 15 etc.
		This again feels impossible. To get an odd multiple of 3
		we would need an even multiple of 3, which we can't do
		using powers of 2.
		However, we can also do numbers of the form $5 \cdot 2^n$.
		We just need any multiple of 3 of the form $5 \cdot 2^n$.
		That also can't exist - 2 and 5 would forever be the only
		prime factors.
		Another step must be required.
	\item I suspect we have to make smart use of doubling.
		It's worth remembering that lone U's can cause us to get
		stuck if we're not careful.
	\item Perhaps we don't need another insight after all, and we
		can reuse the way we managed to get 5.
		We can do any $2^n-3$. That still can't be of the form
		$3n$, though.
	\item There seems to be something going on to do with prime
		numbers - does finding the first nonprime $2^n-3$ help?
		This sequence is $1, 13, 29, 61, 125, 253$.
		The first nonprime is 125. This feels like a dead end
		because we're at a power of 5. I suspect that's all we'll
		get - non-3 primes and their powers.
	\item We can actually get any $2^n-3m$. Does this help?
	\item Let's put that aside for a second, I want to try something.

		MIIIIIU MIIIIIUIIIIIU MIIUUIIUU MIIUUII MIIUUIIIIUUII
	\item Actually, let's create a set of the things that can
		happen numerically to the number of I's.
	\item This is ${\{\times 2, -3\}}$.
	\item This can't ever produce 0, I don't think.
	\item No, it can't. I'll put the proof of this in the section
		below for MU.

\end{itemize}

\paragraph{MI}
This is the axiomatic word - it can simply be asserted.

\paragraph{MIU}
MI (IaddU) MIU

\paragraph{MII}
MI (double) MII

\paragraph{MIIIII}
MI (double) MII (double) MIIII (double) MIIIIIIII 
(IaddU) MIIIIIIIIU (bang) MIIIIIUU (pop) MIIIII

\paragraph{ABC}
This one isn't possible - the only letters that can exist are M, I
(both introduced in the axiom) and U (introduced via IaddU).

\paragraph{MIIUIIU}
(proved above) MIIIII (double) MIIIIIIIIII (bang$\times 2$) MIIUIIU

\paragraph{II}
This one isn't possible - none of the rules allow the M to be removed. 

\paragraph{MU}
This one isn't possible - the number of I's cannot be of the form
$3n$, and therefore cannot be 0. 

\paragraph{Proof:}
The rules have the following effects
on the number of I's:

\begin{itemize}
	\item IaddU: No effect.
	\item Double: $\times 2$.
	\item Bang: $-3$.
	\item Pop: No effect.
\end{itemize}

The initial number of I's is 1, a number of the form $3n+1$.
The double rule will turn any number of the form $3n+1$ into
a number of the form $3n-1$ and vice-versa.
The bang rule will not change the form of a number of the form
$3n+1$ or $3n-1$.

Therefore, the number of I's will always be of the form
$3n+1$ or $3n-1$, and never $3n$.

\subsubsection{Write a program to create all MU system theorems}
Completed. See \nameref{generate_code} for the code.

\subsubsection{Write a program that can test if a string is a
theorem in the MU system}
This is as simple as counting I's and ensuring the count is not
of the form $3n$. Any string with this property can be constructed.

\paragraph{Proof:}

\begin{itemize}
	\item Arbirarily large numbers of consecutive I's
		of the form $2^n$ can be
		created of both the forms $3n+1$ and $3n-1$, because
		powers of 2 always alternate between these two forms.

	\item Arbirarily large numbers of consecutive I's can be
		created of both the forms $3n+1$ and $3n-1$, because you
		can simply create a larger power of 2 of the same form
		and remove any amount of I's of the form $3n$ using the
		sequence (IaddU) (bang) (pop) for each 3 items.
	
	\item Arbitraryly placed U's also pose no problem. Simply
		construct a string with
		$3\text{U}_{count}+\text{I}_{count}$
		I's (this will still be of a correct form mod 3
		since we are adding a multiple of 3).
		Then it's trivial to do the desired (bang) steps.
	
\end{itemize}

See \nameref{validate_code} for the code.

\pagebreak
\begin{thebibliography}{}
	\bibitem{guardianarticle}
‘It’s literally giving somebody money for nothing’: the battle to reform property leaseholds
https://www.theguardian.com/money/2024/feb/10/its-literally-giving-somebody-money-for-nothing-the-battle-to-reform-property-leaseholds
(acessed 10 Feb 2024)
\end{thebibliography}

\section*{Appendix}

\subsection*{main.rs}
\begin{inputminted}{rust}{../../code/01/code/src/main.rs}

\pagebreak
\subsection*{generate.rs}
\label{generate_code}
\begin{inputminted}{rust}{../../code/01/code/src/generate.rs}
\end{inputminted}

\pagebreak
\subsection*{validate.rs}
\label{validate_code}
\begin{inputminted}{rust}{../../code/01/code/src/validate.rs}
\end{inputminted}

\end{document}
