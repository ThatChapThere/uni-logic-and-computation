1. Applications of Lambda Calculus

Boolean Logic

We can define true and false:
T = \xy.x
F = \xy.y

and = \xy.xyF
or = \xy.xTy
not = \x.xFT

examples:

true and true = true
true or false = true
and(or(false, (not true)), true) = false

exercise:

make sure these work
true and true
(\xy.xyF)(\xy.x)(\xy.x)
convert to different symbols
(\xy.xyF)(\ab.a)(\cd.c)
(\ab.a)(\cd.c)F
(\cd.c) = true

true or false
(\xy.xFt)(\ab.a)(\cd.d)
(\ab.a)F(\cd.d)
(\ab.a)(\ef.f)(\cd.d)
(\ef.f) = false

and(or(false, (not true)), true) = false
()

Another nice thing you can do is (for questions):

Q = \bxy.bxy
QTcd = c
QFc(\x.xx) = \x.xx
Q(lots of stuff that = false)cd = d

Church Numerals
0 = \sz.z
1 = \sz.s(z)
2 = \sz.s(s(z))

(mnemonics: s for successor, z for zero)

we can define sucessor function (add one)
S = \wyx.y(wyx)

\wyx = \w.\y.\x. = \w.\yx.

Exercise:
S0
(\wyx.y(wyx))(\sz.z)
(\w.\yx.y(wyx))(\sz.z)
\yx.y((\sz.z)yx)
\yx.y(x) = 1

S2
(\wyx.y(wyx))(\sz(s(s(z))))

Addition:
+ = \ab.aSb
(bear in mind that numbers are also iterators!)

+23 = 2S3 = (\sz.s(s(z))) (\wyx.y(wyx)) (\sz.s(s(s(z))))
= S4 = 5

This constitutes a proof that 2+3=5. This is an example of the duality between programs and proofs.

Multiplication:
* = \xyz.x(yz)
Godel's theorem applies to every logic that can represent addition and multiplication.

Pairs:
represent an ordered pair (a, b)

pair = \abz.zab

pair a b = \z.zab

We can define functions to access elements
first = \p.Tp
second = \p.Fp

Predecessor function:
(copy from notes. no intuitive way of explaining this)

Comparators:

is zero:
Z = \n.nFT
test x >= y
G = \xy.Z(xPy)
test x = y
E = \xy.AND(Gxy)(Gyx) (possible mistake?)

Recursion:

Some expressions don't halt
eg. (\x.xx)(\x.xx)

Consider:
(stuff for Y combinator)

Y can be used to recurse functions of our choice.

(write out arithmetic sum function example)

YR5 = 15

This is the final part that make it Church complete.

2. Meta logic

Meta means something like study from outside, or an application of a subject to itself. From the greek word meaning after/beyond.
Originally from Aristotle's metaphysics, which was just a sequel.
(copy the list of meta things)

The MU system stuff we did was an example of metalogic.

Meta vs Strange Loops:
Strange loops are the strongest form of meta. Meta is typically taking some of X to talk about some other part of X.
Strange loops use some set of somethinng to talk about itself.

Church-Rosser Theorem:
If an expression has a normal form then it doesn't matter what order you do reductions, you'll always arive at the normal form.
This applies to lambda calculus, which means lambda calculus can be easily parallelised.

Halting Problem Proof:

[fill this in]

It's a function constructed specifically to halt if the input does not and vice-versa.
